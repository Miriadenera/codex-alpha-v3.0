% Auto-generated: QC mini-map from qc_minimap_centroids_pG.csv
\pgfplotstableread[col sep=comma]{qc_minimap_centroids_pG.csv}\centroidsA
\begin{figure}[htbp]
  \centering
  \begin{tikzpicture}
    \begin{axis}[
      width=0.70\textwidth, height=0.42\textwidth,
      xlabel={$l$ [deg]}, ylabel={$b$ [deg]},
      grid=both, grid style={black!10},
      enlargelimits=false,
      colormap/viridis,
      colorbar, colorbar style={title={$p_{\mathrm G}$}},
      point meta=explicit,
    ]
      \addplot[
        only marks, mark=*, mark size=1.6pt,
        scatter, scatter src=explicit,
      ]
      table[x=l_mean, y=b_mean, meta=pG]{\centroidsA};
    \end{axis}
  \end{tikzpicture}
  \caption{QC mini-map of node centroids colour-coded by topological persistence on the gradient graph \(G\) (\(p_{\mathrm G}\)). Larger values indicate more robust critical points (deeper basins / higher peaks) with respect to the Morse–Smale decomposition. Axes are Galactic \((l,b)\) in degrees; markers are basin centroids and include all node types.}
  \label{fig:centroidsPersistenceG}
\end{figure}

% Auto-generated: QC mini-map from qc_minimap_centroids_pK.csv
\pgfplotstableread[col sep=comma]{qc_minimap_centroids_pK.csv}\centroidsA
\begin{figure}[htbp]
  \centering
  \begin{tikzpicture}
    \begin{axis}[
      width=0.70\textwidth, height=0.42\textwidth,
      xlabel={$l$ [deg]}, ylabel={$b$ [deg]},
      grid=both, grid style={black!10},
      enlargelimits=false,
      colormap/hot,
      colorbar, colorbar style={title={$p_{\mathrm K}$}},
      point meta=explicit,
    ]
      \addplot[
        only marks, mark=*, mark size=1.6pt,
        scatter, scatter src=explicit,
      ]
      table[x=l_mean, y=b_mean, meta=pK]{\centroidsA};
    \end{axis}
  \end{tikzpicture}
  \caption{QC mini-map of node centroids colour-coded by topological persistence on the scalar field \(K\) (\(p_{\mathrm K}\)). The colourbar reports the unitless persistence used in Sec.~2.2; scales are independent from the \(p_{\mathrm G}\) panel. Axes are Galactic \((l,b)\) in degrees; markers are basin centroids (min/max/saddle).}
  \label{fig:centroidsPersistenceK}
\end{figure}
